\documentclass[12pt]{article}
\usepackage{hyperref}
\usepackage{caption}
\usepackage[margin=1in]{geometry}

\title{Rutgers Room Scheduling Optimization}
\author {William Lynch \\ wlynch92@cs.rutgers.edu \\ Rutgers University
	\and Jingfei Shi \\ jenny.shi@rutgers.edu \\ Rutgers University
}
\begin{document}
\maketitle
\section{Abstract}
An examination of the University Course Timetabling Problem (UCTP) at Rutgers
University is presented. The current scheduling process is a manual process,
which prompts an exploration for ways to algorithmically assign courses so as
to optimize the existing system. The solution features a basic greedy algorithm
to select the best room for a given course. In addition, assigning courses of
the same academic major to rooms in geographic clusters, and the distribution
of introductory courses across sub-campuses are features that can be layered on
top of the greedy algorithm. The results for different ways to schedule courses
are displayed and analyzed. Finally, the various issues and future improvements
are discussed.
\\\\
Keywords: \emph{University Course Timetabling Problem (UCTP), Rutgers University
course scheduling, optimization, course-room assignments, greedy algorithm,
geographical clustering, course distribution}

\section{Introduction}
	Course-room assignment is an integral part of the Rutgers University
	administration. An efficient and functional implementation allows students to
	build schedules with minimal travel time and without conflicts. This is
	especially important for a large, multi-campus university, where students could
	potentially spend a significant amount of time traveling between classes via
	campus buses. While the Scheduling and Space Management office at Rutgers
	(University Scheduling) have created a functional system to provide working
	assignments for each semester, resulting schedules may be inefficient for the
	students. Depending on a student’s academic major, traveling between classes
	located on different campuses can be arduous and seemingly arbitrary (i.e. back
	and forth between two campuses). With that, there is a need to analyze the
	current scheduling process at Rutgers and generate ways to optimize it.

	This process is part of the University Course Timetabling Problem (UCTP). The
	UCTP deals with assigning all events or classes to available time slots and
	classrooms. As described by Murray et al., UCTP is a resource allocation
	problem, where the solution must satisfy both hard and soft constraints so as
	to avoid scheduling conflicts \cite{citation1}. This is largely dependent on the university
	layout, available resources, and main method of transportation. Unfortunately,
	the current scheduling system at Rutgers is mostly a manual process, but this
	also creates room for improvement. 

	This paper provides an overview of the current scheduling implementation at
	Rutgers, as well as existing studies done by researchers at Purdue University.
	Moreover, a solution has been developed for automating and optimizing the
	current system. The basis for this is a greedy algorithm that assigns courses
	to rooms. Rather than over-extending this function, two other factors of the
	scheduling problem are identified and they work in conjunction with the greedy
	algorithm. These factors are concerned with geographical clustering and the
	distribution of introductory classes across the sub-campuses. The algorithm and
	each of these factors will be examined in greater details. Overall, this
	solution successfully schedules 76% of undergraduate lectures and recitations.
	An analysis of the results, as well as factors contributing and/or affecting
	the scheduling process are also detailed in the following sections.

\section{Motivation}
	Although there have been several different methods of scheduling implemented
	for other universities, Rutgers is in a unique position for improvements due to
	its large campus size. Since the New Brunswick campus is spread over a large
	geographic area that is segmented into five sub-campuses, it is difficult to
	make ideal course-room assignments for the entire university. Nevertheless,
	the large campus size highlights the necessity to minimize time spent on the
	campus buses for students. 
	\\\\
	The main motivation behind this optimization is to automate the Rutgers
	scheduling process. At the same time, it endeavors to localize the amount of
	travel needed for students to go between classes. Therefore, it is crucial to
	examine ways to algorithmically assign classes of the same academic major rooms
	in geographic clusters. Both the university and its students may benefit from
	this optimization: while students can enjoy more free time by not having to
	travel back and forth between classes, a reduction in travel means that the
	university does not have to deploy as many buses. This directly translates to
	thousands of dollars worth of savings over time. 

	\subsection{Current Implementation}
		As previously mentioned, the current Rutgers scheduling system is a manual
		process. University Scheduling takes the schedule for the previous semester,
		duplicate the assignments for the courses, and manually fix any conflicts and
		unassigned courses. For example, the Fall 2014 schedule is largely based on the
		Fall 2013 schedule, with minor differences accounting for changes in course
		offering and building availability. While this system is effective in producing
		a working schedule, it does not allow for flexibility of course-room
		assignments. Once a class has been assigned to a certain campus, and if it
		continues to be offered in subsequent semesters without conflicts, its location
		is unlikely to change. However, if an assignment creates a conflict, or if
		there is available space to relocate courses, then the new assignments change
		will persist across future semesters. This gradual shifting makes little sense
		under certain situations, and this directly relates to the motivation for this
		optimization.

		\subsubsection{Course Shifting in the Computer Science Department}
			With the completion of the Rutgers Business School building, most of the
			business classes migrated from the existing classrooms on Livingston to the new
			building. In order to fill in these empty rooms, University Scheduling many
			traditionally “Busch courses” to Livingston. These courses include a majority
			of the Computer Science department. The tables below clearly illustrate this
			shift between the course offering for Fall 2012 and Fall 2014.
			
			\begin{center}
				\captionof{table}{CS Department Courses, Fall 2014} \label{tab:asdf} 
				\begin{tabular}{ | l | l | l | l | }
					\hline
					Course & Campus & Course &	Campus \\ \hline
					111 & CAC, LIV & 344 & LIV \\ \hline
					112 & LIV & 352 & BUS \\ \hline
					205 & LIV & 416 & LIV \\ \hline
					206 & LIV & 417 & LIV \\ \hline
					211 & LIV & 431 & BUS \\ \hline
					214 & LIV & 437 & LIV \\ \hline
					314 & BUS, LIV & 440 & LIV \\ \hline
					334 & BUS & 452 & LIV \\ \hline
					336 & LIV & & \\ 
					\hline
				\end{tabular}
			\end{center}
			\begin{center}
				\captionof{table}{CS Department Courses, Fall 2012} \label{tab:title} 
				\begin{tabular} { | l | l | l | l | }
					\hline
					Course & 	Campus & Course & Campus \\ \hline
					111 & CAC, BUS & 336 & BUS \\ \hline
					112 & LIV, BUS & 344 & BUS \\ \hline
					205 & BUS & 352 & BUS \\ \hline	
					206 & BUS &	416 & BUS \\ \hline
					211 & BUS &	417 & BUS \\ \hline		
					214 & LIV &	431 & BUS \\ \hline		
					314 & BUS & 440 & BUS \\ \hline		
					334 & BUS & & \\
					\hline
				\end{tabular}
			\end{center}
			The newly renovated Livingston classes may have been a good reason to migrate
			the Computer Science courses. However, it poses an inconvenience to the
			students and staff. Since the Computer Science department is on Busch, the
			majority of the professors’ offices are located at Hill Center or CoRE. This
			shift causes an inconvenience for them, since they will have to allocate more
			time for traveling in their schedule in order to teach their courses. For the
			students, their resources (e.g. office hour, peer mentoring, iLab, etc.) are
			all located on Busch. It would make more sense to keep their classes and
			resources on the same campus. 

	\subsection{Goals for Optimization}
	In response to the shortcomings of the current scheduling system, a solution to
	automate the scheduling process has been proposed. In this solution, an “ideal”
	schedule is defined as one where courses in the same department are assigned to
	the same campus. At the same time, the solution should minimize scheduling
	conflicts in an attempt to not overburden a particular campus. 
	\\\\
	Two key factors involved in the UCTP are identified to help produce the said
	schedule. The first factor is the affinity of a course, which dictates that it
	should be assigned near or on the same campus as the location of the
	department. This way, students can minimize travel time while enjoying easy
	access to academic resources for their affiliated departments. The second
	factor is the distribution of introductory courses, where the affinities for
	100 level courses are overruled in favor of equal distribution among all
	campuses. These two factors and their implementations will be further analyzed
	in the Method section.

\section{Existing Research}
The University Course Timetabling Problem has been studied by researchers at
many universities. Specifically, Murray, Müller, and Rudová have provided a
summary and analysis of their scheduling model in [1] under Purdue University
and Masaryk University (Czech Republic). For Murray et al., the UCTP is a
multi-layered problem and can be decomposed into subproblems. In addition,
there exists a set of constraints for each course being scheduled, and these
can be categorized into soft and hard constraints. The solution presented in
this paper is partially influenced by several aspects of their research.

\subsection{Subproblems}
The UCTP is an extremely complex problem, and it is difficult to develop a
solution that will meet all of the details and requirements. With that, Murray
et al. approached the problem by breaking it into a general course model with
customizable constraints and relationships. This enables the use of a single
solution that will apply to all courses, instead of separate designed
specifically to fit certain departments. Each aspect of this model can then be
broken down into subproblems with specific characteristics. Solutions to each
subproblem are found and combined into one complete implementation.
\\\\
Some subproblems identified by Murray et al. are listed below:

\begin{enumerate}
	\item centrally timetabled large lecture room problem
	\item individually timetabled departmental problems
	\item centrally timetabled computer laboratory problem
\end{enumerate}

While the timetabling system is different at Purdue University, the general
idea of decomposition is applicable for Rutgers University scheduling. All
courses have lectures, and some include recitations and labs. The proposed
solution only analyzes lectures and recitations, but the subproblem approach is
largely inspired by Murray et al.

\subsection{Soft vs. Hard Constraints}
In addition to being a resource allocation problem, the UCTP problem can also
be viewed as a constraint satisfaction and optimization problem (CSOP). Each
course being scheduled contains a set of finite constraints, which can be
categorized into hard or soft. Hard constraints are constraints that must be
met in order to successfully schedule a course, while soft constraints are
associated with optimization. Therefore, the goal of the constraint
satisfaction and optimization problem is to meet all of the hard constraints,
while violating the least number of soft constraints.
\\\\
Recognizing the UCTP as a CSOP is the driving force behind the design to this
proposed optimization. It is obvious that courses must be scheduled based on
their capacity and time--the hard constraints. The soft constraints aspect of
the CSOP inspired the addition of two factors: affinity and distribution of
introductory courses. This solution endeavors to satisfy these soft constraints
after fulfilling the hard constraints.

\section{Method}
	\subsection{Generate Generic Rooms}
	
	\subsection{Best-Fit}

	\subsection{Geographical Clustering}

	\subsection{Distribution of Introductory Courses}

\section{Experiment}

	\subsection{Data Collection and Organization}

	\subsection{Data Separation}

\section{Results}

	\subsection{Sample Runs}
		Number of generic rooms: 10 \\
		Generic room capacity: 40
		\begin{center}
			\captionof{table}{Greedy Algorithm} \label{tab:title} 
			\begin{tabular} { | l | l | l | l | l | l | l | l | l | }
				\hline
				Dept & AFF & BUS & LIV & CAC & D/C & DNB & Other & Total \\ \hline
				165 & CAC & 0 & 0 & 37 & 0 & 0 & 0 & 37 \\ \hline
				198 & BUS & 44 & 2 & 4 & 1 & 0 & 0 & 51 \\ \hline
				355 & LIV & 111 & 184 & 196 & 185 & 41 & 0 & 717 \\ \hline
				640 & BUS & 123 & 132 & 55 & 70 & 9 & 0 & 389 \\ \hline
				940 & D/C & 2 & 11 & 15 & 50 & 1 & 0 & 79 \\ \hline
			\end{tabular}
		\end{center}

		\begin{center}
			\captionof{table}{Affinity} \label{tab:title} 
			\begin{tabular} { | l | l | l | l | l | l | l | l | l | }
				\hline
				Dept & AFF & BUS & LIV & CAC & D/C & DNB & Other & Total \\ \hline
				165 & CAC & 0 & 0 & 37 & 0 & 0 & 0 & 37 \\ \hline
				198 & BUS & 44 & 1 & 5 & 1 & 0 & 0 & 51 \\ \hline
				355 & LIV & 93 & 226 & 179 & 179 & 40 & 0 & 717 \\ \hline
				640 & BUS & 155 & 129 & 36 & 64 & 5 & 0 & 389 \\ \hline
				940 & D/C & 0 & 6 & 17 & 56 & 0 & 0 & 79 \\ \hline
			\end{tabular}
		\end{center}

		\begin{center}
			\captionof{table}{Distribution} \label{tab:title} 
			\begin{tabular} { | l | l | l | l | l | l | l | l | l | }
				\hline
				Dept & AFF & BUS & LIV & CAC & D/C & DNB & Other & Total \\ \hline
				165 & CAC & 0 & 0 & 37 & 0 & 0 & 0 & 37 \\ \hline
				198 & BUS & 44 & 2 & 4 & 1 & 0 & 0 & 51 \\ \hline
				355 & LIV & 111 & 184 & 196 & 185 & 41 & 0 & 717 \\ \hline
				640 & BUS & 123 & 132 & 54 & 70 & 10 & 0 & 389 \\ \hline
				940 & D/C & 2 & 11 & 16 & 50 & 0 & 0 & 79 \\ \hline
			\end{tabular}
		\end{center}

		\begin{center}
			\captionof{table}{Affinity and Distribution} \label{tab:title} 
			\begin{tabular} { | l | l | l | l | l | l | l | l | l | }
				\hline
				Dept & AFF & BUS & LIV & CAC & D/C & DNB & Other & Total \\ \hline
				165 & CAC & 0 & 0 & 37 & 0 & 0 & 0 & 37 \\ \hline
				198 & BUS & 44 & 1 & 5 & 1 & 0 & 0 & 51 \\ \hline
				355 & LIV & 93 & 226 & 178 & 179 & 41 & 0 & 717 \\ \hline
				640 & BUS & 152 & 129 & 35 & 64 & 9 & 0 & 389 \\ \hline
				940 & D/C & 0 & 6 & 17 & 56 & 0 & 0 & 79 \\ \hline
			\end{tabular}
		\end{center}

		Total number of classes: 6522 \\
		Successfully scheduled: 4940 \\
		Failed to schedule: 1582 \\

	\subsection{Analysis}

	\subsection{Possible Explanations}

		\subsubsection{Affinity Assignment}

		\subsubsection{Strain of Resources}

\section{Issues}

	\subsection{Pooly Structure Data}
	
	\subsection{Incomplete Data}

\section{Conclusions}

\section{Acknowledgement}

\section{References}
\begin{thebibliography}{9}
	\bibitem{citation1}
		Keith Murray, Tomáš Müller, and Hana Rudová.
		Modeling and Solution of a Complex University Course Timetabling Problem.
		In Edmund Burke and Hana Rudová, editors,
		\emph{Practice and Theory of Automated Timetabling, Selected Revised Papers},
		Springer-Verlag LNCS 3867,
		Pages 189-209,
		2007.
\end{thebibliography}
\end{document}
